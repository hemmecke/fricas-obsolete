\documentclass[twoside,a4paper]{book}
\usepackage{fricas}

\begin{document}
%
% \thispagestyle{empty}
% \vspace*{1in}
% {\Large\sf Richard D. Jenks \hfill Robert S. Sutor \newline
% \vskip 1in
% \begin{center}
% \LARGE\sf AXIOM \\
% \quad \\ \quad \\
% \Large\sf The Scientific Computation System
% \quad \\ \quad \\ \vskip 1.5in
% \normalsize\rm Draft: \today
% \end{center}}
% \newpage
% \thispagestyle{empty}
% \quad
%
\pagenumbering{roman}
\setcounter{page}{0}

\begin{quote}
\color{red}\Large
  This is an attempt to make the content of the Axiom book (by Jenks
  and Sutor) available again in order to describe the FriCAS project
  (which is a fork of the original Axiom code). The following
  material is mainly taken from the original Axiom book, but is partly
  tailored to new developments in FriCAS.

  \vspace{1cm}

  \textbf{WARNING: This is work-in-progress! There might be errors and
    even false statements.}

  Plan is as follows:
  \begin{enumerate}
  \item Make LaTeX compilation work and include (generated) pictures
    and generated output of algebra commands.
  \item Simplify and remove redundant \LaTeX{} commands.
  \item Build on modern advanced latex packages.
  \item Generate indices and hyperrefs.
  \end{enumerate}

  All of the above will be done by taking care of still generating the
  same \texttt{.ht} files although the technicalities might look
  different. The contents of the files (no matter how wrong it
  actually is) will only be changed marginally. Rewriting of the
  contents will only happen after the technical part of producing the
  PDF file has been finished.

  \hspace*{\fill}Ralf Hemmecke
\end{quote}

%\input{dedicate}
%\input{foreword}
\clearpage
\tableofcontents
%\input{bios}
%
\pagenumbering{arabic}
\setcounter{chapter}{-1}
\input{ug00}
\input{tecintro}  % This ha no chapter number.

\part{Basic Features of \Language{}}
%
\input{ug01}
\input{ug02}
\input{ug03}
\input{ug04}
\input{ug05}
\input{ug06}

%--rhx: TODO:
% Unfortunately ug07 uses \item[\spadfun{foo}] which doesn't work
% since \spadfun has a verbatim-like argument. Fortunately, that only
% happens inside the description environment. So we simply redefine it
% for this chapter. Better rewrite the text in ug07.
\begingroup
\let\spaddescriptionsave\description
\def\description{\def\spadfun{\textspadfun}\spaddescriptionsave}
\input{ug07}
\endgroup

\part{Advanced Problem Solving and Examples}
%
\input{ug08}
\input{ug09}

\part{Advanced Programming in \Language{}}
%
\input{ug10}
\input{ug11}
\input{ug12}
\input{ug13}
\input{ug14}
\input{ug15}
%
\appendix

\input{ug16}
%\input{ug17}
%\input{ug18}
%\input{ug19}
%\input{ug20}
\input{ug21}
%
\printindex
\end{document}
